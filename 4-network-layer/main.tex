Netværkslaget har ansvaret for at rute diagrammer fra en vært til en anden

\section{Introduktion}
\begin{itemize}
	\item host-to-host kommmunikation
	\item forbindelsesorienteret (eks. telefonsamtale) - pålidelig
	\item forbindelsesløs - upålidelig
	\item videresendelse
	\item stibestemmelse
	\item connection setup
\end{itemize}

\section{Netværk service modeller}
\begin{itemize}
	\item guaranteed delivery
	\item guaranteed delivery with bounded delay
	\item Internettet
	\item Constant bitrate (CBR)
	\item Available Bit Rate (ABR)
\end{itemize}

\section{Pakkekoblede netværk}
\begin{itemize}
	\item datagramnetværk 
	\item virtuelle kredsløbs koblede netværk
	\item Internettet
	\item Constant bitrate (CBR)
	\item Available Bit Rate (ABR)
\end{itemize}

\subsection{Virtuelle kredsløbs koblede netværk}
\begin{itemize}
	\item rute 
	\item VC nummer
	\item videresendelsestabel
	\item 3 faser - VC setup - Data transfer - VC teardown
	\item opbevarer oplysninger om forbindelsesstatus
	\item videresendes ud fra den forbindelsen pakken tilhører
\end{itemize}

\subsection{Datagramnetværk}
\begin{itemize}
	\item kun kendt for de to slutsystemer
	\item header (hierarkisk sturktur) 
	\item videresendelsestabel
	\item eks. med postvæsenet
	\item ikke garanti for rigtig rækkefølge
	\item opbevarer INGEN oplysninger om forbindelsesstatus
	\item videresendes ud fra modtageradressen
\end{itemize}

\section{Routingprincipper}
Bestemmelsen af routen er en opgave der tilhører routingprotokollen, eller nærmere bestemt den routingalgoritme der er en del af routingprotokollen.
\begin{itemize}
	\item transportere datagrammer
	\item bestemmelse af route (routingprotokollen)
	\item Global routingalgoritme – Link state algoritmer
	\item Decentral routingalgoritme – Distance vektor algoritmer
\end{itemize}

\subsection{Global routingalgoritme – Link state algoritmer}
\begin{itemize}
	\item Den billigste sti mellem afsender og modtager
	\item global
	\item broadcast
	\item routing tabel
	\item dynamisk
\end{itemize}

\subsection{Decentral routingalgoritme – Distance vektor algoritmer}
\begin{itemize}
	\item Den hurtigste sti fra afsender til modtager
	\item iterativ og distribueret
	\item ingen kendskab til omkostningerne i netværket
	\item naboer
	\item kender ikke den komplette sti
	\item statisk
	\item asynkron
	\item distancetabel
\end{itemize}

\section{Sådan virker en router}
En router er en enhed på et computernetværk som forbinder et antal logiske eller fysiske netværk ved at videresende pakker fra et netværk til deres destination på et andet netværk i en process kaldet routing. Routeren arbejder på OSI-modellens lag 3 (netværkslaget), i internetprotokollen kaldes dette lag IP-laget.
\begin{itemize}
	\item input ports
	\item switching fabric
	\item output ports
	\item routing processor
	\item packet scheduler(opfyldelsesgaranti, isolere 'flows' fra hinanden)
	\item packet scheduler algoritmer(FCFS - WFQ - RR)
	\item QoS
\end{itemize}

\section{Hierarkisk routning}
En router er en enhed på et computernetværk som forbinder et antal logiske eller fysiske netværk ved at videresende pakker fra et netværk til deres destination på et andet netværk i en process kaldet routing. Routeren arbejder på OSI-modellens lag 3 (netværkslaget), i internetprotokollen kaldes dette lag IP-laget.
\begin{itemize}
	\item autonome systemer(AS)
	\item klassificeret i regioner(reducere hukommelses krav)
	\item intraautonom system routing protokol
	\item hot-potato routing(hurtigt og lave omkostninger - pakke => gatewayrouter)
\end{itemize}

\section{Internetprotokollen – IP}
En router er en enhed på et computernetværk som forbinder et antal logiske eller fysiske netværk ved at videresende pakker fra et netværk til deres destination på et andet netværk i en process kaldet routing. Routeren arbejder på OSI-modellens lag 3 (netværkslaget), i internetprotokollen kaldes dette lag IP-laget.
\begin{itemize}
	\item autonome systemer(AS)
	\item klassificeret i regioner(reducere hukommelses krav)
	\item intraautonom system routing protokol
	\item hot-potato routing(hurtigt og lave omkostninger - pakke => gatewayrouter)
\end{itemize}